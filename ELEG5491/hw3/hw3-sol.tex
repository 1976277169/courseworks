\documentclass{article}

\usepackage{fancyhdr} % Required for custom headers
\usepackage{lastpage} % Required to determine the last page for the footer
\usepackage{amsmath}
\usepackage{amsfonts}
\usepackage{dsfont}
\usepackage{graphicx}
\usepackage{subfigure}
\usepackage{listings}
\usepackage{booktabs}
\usepackage{mmstyle}
\usepackage[hidelinks]{hyperref}
% \usepackage{epstopdf} uncomment this line if using MiKTeX

% Margins
\topmargin=-0.45in
\evensidemargin=0in
\oddsidemargin=0in
\textwidth=6.5in
\textheight=9.0in
\headsep=0.25in

\linespread{1.25} % Line spacing

% Set up the header and footer
\pagestyle{fancy}
\lhead{\authorName} % Top left header
\chead{\classID\ \hwTitle} % Top center header
\rhead{\studentID} % Top right header
\lfoot{} % Bottom left footer
\cfoot{} % Bottom center footer
\rfoot{Page\ \thepage\ of~\pageref{LastPage}} % Bottom right footer
\renewcommand{\headrulewidth}{0.4pt} % Size of the header rule
\renewcommand{\footrulewidth}{0.4pt} % Size of the footer rule

\setlength{\parindent}{0pt} % Removes all indentation from paragraphs

\setcounter{secnumdepth}{0} % Removes default section numbers
\newcounter{problemCounter} % Creates a counter to keep track of the number of problems

\newcommand{\problemName}{}
\newenvironment{problem}[1][Problem \arabic{problemCounter}]{
	\stepcounter{problemCounter} % Increase counter for number of problems
	\renewcommand{\problemName}{#1} % Assign \problemName the name of the problem
	\section{\problemName} % Make a section in the document with the custom problem count
}{}

\newcommand{\subproblemName}{}
	\newenvironment{subproblem}[1]{
	\renewcommand{\subproblemName}{#1} % Assign \subproblemName to the name of the section from the environment argument
	\subsection{\subproblemName} % Make a subsection with the custom name of the subsection
}{}

\newcommand{\norm}[1]{\left\lVert#1\right\rVert}

%----------------------------------------------------------------------------------------
%	NAME AND CLASS SECTION
%----------------------------------------------------------------------------------------

\newcommand{\hwTitle}{Assignment\ \#3} % Assignment title
\newcommand{\dueDate}{Thursday,\ April\ 6,\ 2017} % Due date
\newcommand{\classID}{ELEG\ 5491} % Course/Class
\newcommand{\authorName}{Kai Chen} % Your name
\newcommand{\studentID}{1155070509} % Your student ID

%----------------------------------------------------------------------------------------
%	TITLE PAGE
%----------------------------------------------------------------------------------------

\title{
	\vspace{2in}
	\textmd{\textbf{\classID:\ \hwTitle}}\\
	\normalsize\vspace{0.1in}\small{Due\ on\ \dueDate}
	\vspace{3in}
}

\author{\textbf{\authorName}}
\date{} % Insert date here if you want it to appear below your name

%----------------------------------------------------------------------------------------

\begin{document}
\maketitle
\setcounter{page}{0}
\thispagestyle{empty}
\newpage

%----------------------------------------------------------------------------------------
%	PROBLEM 1
%----------------------------------------------------------------------------------------

% To have just one problem per page, simply put a \clearpage after each problem

\begin{problem}

\begin{subproblem}{1.1}

\begin{align*}
P(x) &= \sum_\vh P(\vx,\vh) \\
&= \sum_\vh \frac{e^{-E(\vx,\vh)}}{Z} \\
&= \frac{e^{-\cF(x)}}{Z}
\end{align*}
where
\[
Z = \sum_{\vx,\vh}e^{-E(\vx,\vh)} = \sum_\vx e^{-\cF{(x)}}
\]

\end{subproblem}

\begin{subproblem}{1.2}

\begin{align*}
-\frac{\partial\log p(\vx)}{\partial\theta} &= -\frac{\partial}{\partial\theta}\frac{e^{-\cF(\vx)}}{\sum_{\hat{\vx}}e^{-\cF(\hat{\vx})}} \\
&= -\frac{\partial}{\partial\theta}(-\cF(\vx) - \log\sum_{\hat{\vx}}e^{-\cF(\hat{\vx})}) \\
&= \frac{\partial\cF(\vx)}{\partial\theta} - \sum_\vx \frac{e^{-\cF(\hat{\vx})}}{\sum_{\hat{\vx}}e^{-\cF(\hat{\vx})}}\frac{\partial\cF(\hat{\vx})}{\partial\theta} \\
&= \frac{\partial\cF(\vx)}{\partial\theta} -\sum_\vx p(\hat{\vx})\frac{\partial\cF(\hat{\vx})}{\partial\theta}
\end{align*}

\end{subproblem}

\begin{subproblem}{1.3}

If the model is RBM, $\{h_i\}$ are conditionally independent given $\vx$, $\{x_i\}$ are also  conditionally independent given $\vh$.
\begin{align*}
P(\vh|\vx) &= \prod_i P(h_i | \vx) \\
P(\vx|\vh) &= \prod_i P(x_i | \vh)
\end{align*}
If all neurons are binary and activation is sigmoid, the conditional distributions have analytical solutions. To generate samples from the model distribution, we can use MCMC.

The special structure of RBM ensures the conditionally independency, so that the sampling of $h_i^{(n+1)}$ only depends on $x^n$ other than $h_j^{(n+1)}$, block Gibbs sampling can be performed.

\end{subproblem}

\end{problem}

%----------------------------------------------------------------------------------------
%	PROBLEM 2
%----------------------------------------------------------------------------------------

\begin{problem}
	
The goal of PCA is to minimize
\[
J = \frac{1}{N}\sum_{n=1}^N \norm{\vx_n - \tilde{\vx}_n} 
\]
where $\tilde{\vx}_n$ is the reconstructed vector of projection of $\vx_n$. And this is exactly the target of a linear autoencoder.


\end{problem}

\end{document}
