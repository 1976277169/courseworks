\documentclass{article}

\usepackage{fancyhdr} % Required for custom headers
\usepackage{lastpage} % Required to determine the last page for the footer
\usepackage{amsmath}
\usepackage{amsfonts}
\usepackage{graphicx}
\usepackage{listings} 
\usepackage[hidelinks]{hyperref}
% \usepackage{epstopdf} uncomment this line if using MiKTeX

% Margins
\topmargin=-0.45in
\evensidemargin=0in
\oddsidemargin=0in
\textwidth=6.5in
\textheight=9.0in
\headsep=0.25in 

\linespread{1.25} % Line spacing

% Set up the header and footer
\pagestyle{fancy}
\lhead{\authorName} % Top left header
\chead{\classID\ \hwTitle} % Top center header
\rhead{\studentID} % Top right header
\lfoot{} % Bottom left footer
\cfoot{} % Bottom center footer
\rfoot{Page\ \thepage\ of~\pageref{LastPage}} % Bottom right footer
\renewcommand{\headrulewidth}{0.4pt} % Size of the header rule
\renewcommand{\footrulewidth}{0.4pt} % Size of the footer rule

\setlength{\parindent}{0pt} % Removes all indentation from paragraphs

\setcounter{secnumdepth}{0} % Removes default section numbers
\newcounter{problemCounter} % Creates a counter to keep track of the number of problems

\newcommand{\problemName}{}
\newenvironment{problem}[1][Problem \arabic{problemCounter}]{
	\stepcounter{problemCounter} % Increase counter for number of problems
	\renewcommand{\problemName}{#1} % Assign \problemName the name of the problem
	\section{\problemName} % Make a section in the document with the custom problem count
}{}

\newcommand{\subproblemName}{}
	\newenvironment{subproblem}[1]{
	\renewcommand{\subproblemName}{#1} % Assign \subproblemName to the name of the section from the environment argument
	\subsection{\subproblemName} % Make a subsection with the custom name of the subsection
}{}

%----------------------------------------------------------------------------------------
%	MATH OPERATOR
%----------------------------------------------------------------------------------------

\DeclareMathOperator*{\argmin}{arg\,min}
\DeclareMathOperator*{\argmax}{arg\,max}

%----------------------------------------------------------------------------------------
%	NAME AND CLASS SECTION
%----------------------------------------------------------------------------------------

\newcommand{\hwTitle}{Assignment\ \#2} % Assignment title
\newcommand{\dueDate}{Thursday,\ March\ 10,\ 2015} % Due date
\newcommand{\classID}{ENGG\ 5202} % Course/Class
\newcommand{\authorName}{Kai Chen} % Your name
\newcommand{\studentID}{1155070509} % Your student ID

%----------------------------------------------------------------------------------------
%	TITLE PAGE
%----------------------------------------------------------------------------------------

\title{
	\vspace{2in}
	\textmd{\textbf{\classID:\ \hwTitle}}\\
	\normalsize\vspace{0.1in}\small{Due\ on\ \dueDate}
	\vspace{3in}
}

\author{\textbf{\authorName}}
\date{} % Insert date here if you want it to appear below your name

%----------------------------------------------------------------------------------------

\begin{document}
\maketitle
\setcounter{page}{0}
\thispagestyle{empty}
\newpage

%----------------------------------------------------------------------------------------
%	PROBLEM 1
%----------------------------------------------------------------------------------------

% To have just one problem per page, simply put a \clearpage after each problem

\begin{problem}

\begin{subproblem}{1.1}

\begin{align*}
	l(\theta_1) &= \log p(D_1|\omega_1,\theta_1) \\
	&= \log p(x_{11}|\omega_1,\theta_1) + \log p(x_{12}|\omega_1,\theta_1) \\
	&= 2\log \theta_1 - \theta_1(x_{11}+x_{12})
\end{align*}
Let
\[
	\nabla l(\theta_1) = 0
\]
We get
\[
	\hat{\theta_1} = \frac{2}{x_{11}+x_{12}} = \frac{1}{3}
\]
Similarly,
\[
	\hat{\theta_2} = \frac{2}{x_{21}+x_{22}} = \frac{1}{6}
\]

\end{subproblem}

\begin{subproblem}{1.2}

Given \(\theta_1 = \frac{1}{3}\) and \(\theta_2 = \frac{1}{6}\), we have \\
\[
p(x|\omega_1) = 
	\begin{cases}
		0 & x < 0 \\
		\frac{1}{3}\mathrm{e}^{-\frac{x}{3}} & x\geq0
	\end{cases}
\]
\[
p(x|\omega_2) = 
	\begin{cases}
		0 & x < 0 \\
		\frac{1}{6}\mathrm{e}^{-\frac{x}{6}} & x\geq0
	\end{cases}
\]
\begin{align*}
	g(x) &= p(\omega_1|x) - p(\omega_2|x) \\
	&= \ln\frac{p(x|\omega_1)}{p(x|\omega_2)} - \ln\frac{p(\omega_1)}{p(\omega_2)}
\end{align*}
Let
\[
	g(x) = 0
\]
We have
\begin{align*}
	x^* &= \frac{\ln\theta_1-\ln\theta_2}{\theta_1-\theta_2} \\
	&= 6\ln2\approx4.16
\end{align*}

\end{subproblem}

\begin{subproblem}{1.3}

The classification rule is when \(x>x^*\), class label is 2, when \(0<x<x^*\), class label is 1. \\
So the expected classification error
\begin{align*}
	Error &= \int_0^{x^*}p(x|\omega_2)p(\omega_2)\mathrm{d}x + \int_{x^*}^\infty p(x|\omega_1)p(\omega_1)\mathrm{d}x \\
	&= \frac{1}{2}\int_0^{6\ln2}\mathrm{e}^{-\frac{x}{6}}\mathrm{d}x + \frac{1}{2}\int_{6\ln2}^\infty\frac{1}{3}\mathrm{e}^{-\frac{x}{3}}\mathrm{d}x \\
	&= \frac{3}{8}
\end{align*}

\end{subproblem}

\end{problem}

%----------------------------------------------------------------------------------------
%	PROBLEM 2
%----------------------------------------------------------------------------------------

\begin{problem}

\(a_{12}\) is updated in the M step and
\begin{align*}
\hat{a}_{12} &= \frac{\sum_{t=2}^T\xi_{t-1}(\omega_1,\omega_2)}{\sum_{t=2}^T\sum_{j'=1}^c\xi_{t-1}(\omega_1,\omega_{j'})} \\
&= \frac{\sum_{t=2}^T P(Z_{t-1}=\omega_1,Z_t=\omega_2|X,\theta^{old})}{\sum_{t=2}^T\sum_{j'=1}^c\xi_{t-1}(\omega_1,\omega_{j'})}
\end{align*}
Because \(a_{12}\) is initialized as 0, therefore for any \(t>1\),we have
\[
	P(Z_{t-1}=\omega_1,Z_t=\omega_2|X,\theta^{old}) = 0
\]
So in all subsequence updates of the EM algorithm \(\hat{a}_{12}=0\).
\end{problem}

%----------------------------------------------------------------------------------------
%	PROBLEM 3
%----------------------------------------------------------------------------------------

\begin{problem}

\begin{subproblem}{3.1}
The sample size for the ``same density'' in \(\mathcal{R}^d\) is \(\sqrt[d]{n_1}\). \\
If \(n_1=100\), the sample size needed is \(100^{20}\).
\end{subproblem}

\begin{subproblem}{3.2}

\end{subproblem}

\end{problem}

%----------------------------------------------------------------------------------------
%	PROBLEM 4
%----------------------------------------------------------------------------------------

\begin{problem}



\end{problem}

\end{document}
